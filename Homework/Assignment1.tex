\documentclass[12pt]{article}

% -----------------------------
% Packages
% -----------------------------
\usepackage[utf8]{inputenc}
\usepackage[T1]{fontenc}
\usepackage{lmodern}
\usepackage[most]{tcolorbox}
\usepackage{amsmath, amssymb, amsthm}
\usepackage{mathtools}
\usepackage{physics}          % Derivatives, vector notation, etc.
\usepackage{bm}               % Bold math symbols
\usepackage{cancel}           % For striking out terms
\usepackage{xcolor}           % For highlighting
\usepackage[dvipsnames]{xcolor}
\usepackage{tikz}             % Diagrams
\usepackage{tikz-3dplot}
\tdplotsetmaincoords{70}{110}
\usepackage{enumitem}         % Customizable lists
\usepackage{hyperref}         % Clickable references
\usepackage{geometry}         % Page margins

% -----------------------------
% Geometry
% -----------------------------
\geometry{margin=1in}

% -----------------------------
% Theorem / Definition Styles
% -----------------------------
\newtheorem{theorem}{Theorem}[section]
\newtheorem{lemma}[theorem]{Lemma}
\newtheorem{proposition}[theorem]{Proposition}
\newtheorem{corollary}[theorem]{Corollary}
\newtheorem{definition}[theorem]{Definition}
\newtheorem{example}[theorem]{Example}
\newtheorem{remark}[theorem]{Remark}

% -----------------------------
% Common Sets
% -----------------------------
\newcommand{\R}{\mathbb{R}}
\newcommand{\N}{\mathbb{N}}
\newcommand{\Z}{\mathbb{Z}}
\newcommand{\Q}{\mathbb{Q}}
\newcommand{\C}{\mathbb{C}}

% -----------------------------
% Linear Algebra Shortcuts
% -----------------------------
\newcommand{\vect}[1]{\mathbf{#1}}          % Vectors in bold
\newcommand{\mat}[1]{\mathbf{#1}}           % Matrices in bold
\newcommand{\inv}{^{-1}}                    % Inverse
\DeclareMathOperator{\Span}{span}
\DeclareMathOperator{\nullity}{nullity}
\DeclareMathOperator{\diag}{diag}
\DeclareMathOperator{\Det}{det}

% -----------------------------
% Differential Equations Shortcuts
% -----------------------------
\newcommand{\ddx}{\frac{d}{dx}}
\newcommand{\ddy}[1]{\frac{d}{d#1}}
\newcommand{\ppx}{\frac{\partial}{\partial x}}
\newcommand{\ppy}[1]{\frac{\partial}{\partial #1}}

% Dot notation for ODEs
\newcommand{\dotx}{\dot{x}}
\newcommand{\doty}{\dot{y}}
\newcommand{\dotz}{\dot{z}}

% -----------------------------
% Hyperref Setup
% -----------------------------
\hypersetup{
    colorlinks = true,
    linkcolor  = blue,
    urlcolor   = teal,
    citecolor  = magenta
}

% -----------------------------
% Stylistic Options and Aux Settings
% -----------------------------

\setlength{\parindent}{0pt}
\allowdisplaybreaks
\tcbset{problem/.style={colback=Gray!20, colframe=Black, breakable, enhanced jigsaw, arc=0.5mm, boxrule=0.5pt}}


% -----------------------------
% Document Info
% -----------------------------
\title{Homework 1}
\author{Wilson Chen}
\date{August 27, 2025}

\begin{document}

\maketitle

\textbf{Problem 1.1.8:} \begin{align*}
    \begin{bmatrix}
        1 & 1 & 2 & 0 \\
        0 & 1 & 7 & 0 \\
        0 & 0 & 2 & -2
    \end{bmatrix} 
    \xrightarrow{\; R_3 \to \frac{1}{2}R_3 \;}
    \begin{bmatrix}
        1 & 1 & 2 & 0 \\
        0 & 1 & 7 & 0 \\
        0 & 0 & 1 & -1
    \end{bmatrix}
\end{align*}\begin{align*}
    & \therefore x_3 = -1 \\[6pt]
    & x_2 + 7x_3 = 0 \therefore x_2 = 7 \\[6pt]
    & x_1 + x_2 + 2x_3 = 0 \therefore x_1 = -5 \\[6pt]
    & \boxed{(x_1, x_2, x_3) = (-5, 7, -1)}
\end{align*}

\newpage

\textbf{Problem 1.1.12:} \begin{align*}
    \begin{bmatrix}
        1 & -3 & 4 & -4 \\
        3 & -7 & 7 & -8 \\
        -4 & 6 & 2 & 4
    \end{bmatrix}
\end{align*} \begin{align*}
    \begin{bmatrix}
        1 & -3 & 4 & -4 \\
        3 & -7 & 7 & -8 \\
        -4 & 6 & 2 & 4
    \end{bmatrix}
    & \xrightarrow{\; R_3 \to R_1 + R_2 \;}
    \begin{bmatrix}
        1 & -3 & 4 & -4 \\
        3 & -7 & 7 & -8 \\
        0 & -4 & 13 & -8
    \end{bmatrix} \\[6pt]
    \begin{bmatrix}
        1 & -3 & 4 & -4 \\
        3 & -7 & 7 & -8 \\
        0 & -4 & 13 & -8
    \end{bmatrix} 
    & \xrightarrow{\; R_2 \to R_2 + 3R_1 \;}
    \begin{bmatrix}
        1 & -3 & 4 & -4 \\
        0 & 2 & -5 & 4 \\
        0 & -4 & 13 & -8
    \end{bmatrix} \\[6pt]
    \begin{bmatrix}
        1 & -3 & 4 & -4 \\
        0 & 2 & -5 & 4 \\
        0 & -4 & 13 & -8
    \end{bmatrix}
    & \xrightarrow{\; R_3 \to R_3 + 2R_2 \;}
    \begin{bmatrix}
        1 & -3 & 4 & -4 \\
        0 & 2 & -5 & 4 \\
        0 & 0 & 3 & 0
    \end{bmatrix}
\end{align*} \begin{align*}
    & \therefore x_3 = 0 \\[6pt]
    & 2x_2 - 5x_3 = 4 \therefore x_2 = 2 \\[6pt]
    & x_1 - 3x_2 + 4x_3 = -4 \therefore x_1 = 2 \\[6pt]
    & \boxed{(x_1, x_2, x_3) = (2, 2, 0)}
\end{align*}

\newpage

\textbf{Problem 1.1.20} \begin{align*}
    \begin{bmatrix}
        1 & 0 & 0 & -2 & -3 \\
        0 & 2 & 2 & 0 & 0 \\
        0 & 0 & 1 & 3 & 1 \\
        -2 & 3 & 2 & 1 & 5
    \end{bmatrix}
\end{align*} \begin{align*}
    \begin{bmatrix}
        1 & 0 & 0 & -2 & -3 \\
        0 & 2 & 2 & 0 & 0 \\
        0 & 0 & 1 & 3 & 1 \\
        -2 & 3 & 2 & 1 & 5
    \end{bmatrix}
    & \xrightarrow{\; R_1 \leftrightarrow R_4 \;}
    \begin{bmatrix}
        1 & 0 & 0 & -2 & -3 \\
        0 & 2 & 2 & 0 & 0 \\
        0 & 0 & 1 & 3 & 1 \\
        -2 & 3 & 2 & 1 & 5
    \end{bmatrix} \\[6pt]
    \begin{bmatrix}
        1 & 0 & 0 & -2 & -3 \\
        0 & 2 & 2 & 0 & 0 \\
        0 & 0 & 1 & 3 & 1 \\
        -2 & 3 & 2 & 1 & 5
    \end{bmatrix}
    & \xrightarrow{\; R_4 \to R_4 + 2R_1 + R_3 \;}
    \begin{bmatrix}
        1 & 0 & 0 & -2 & -3 \\
        0 & 2 & 2 & 0 & 0 \\
        0 & 0 & 1 & 3 & 1 \\
        0 & 3 & 3 & 0 & 0 \\
    \end{bmatrix} \\[6pt]
    \begin{bmatrix}
        1 & 0 & 0 & -2 & -3 \\
        0 & 2 & 2 & 0 & 0 \\
        0 & 0 & 1 & 3 & 1 \\
        0 & 3 & 3 & 0 & 0 \\
    \end{bmatrix} 
    & \xrightarrow{\; R_4 \to R_4 - \frac{3}{2}R_2 \;}
    \begin{bmatrix}
        1 & 0 & 0 & -2 & -3 \\
        0 & 2 & 2 & 0 & 0 \\
        0 & 0 & 1 & 3 & 1 \\
        0 & 0 & 0 & 0 & 0 \\
    \end{bmatrix} 
\end{align*}
This linear system is $\boxed{\text{consistent}}$ because the augmented matrix has no row of the form $[0 \;\; \cdots \;\; 0, \; b]$ where $b$ is a non-zero number. \par

\newpage

\textbf{Problem 1.1.24} \begin{align*}
    \begin{bmatrix}
        1 & h & -3 \\
        -2 & 4 & 6 \\
    \end{bmatrix}
\end{align*} \begin{align*}
    \begin{bmatrix}
        1 & h & -3 \\
        -2 & 4 & 6 \\
    \end{bmatrix} 
    & \xrightarrow{\; R_1 \to R_1 + \frac{1}{2}R_2 \;}
    \begin{bmatrix}
        0 & 2h & 0 \\
        -2 & 4 & 6 \\
    \end{bmatrix} 
\end{align*}
This linear system is consistent for $\boxed{h \in \R}$ because the augmented matrix will have no row of the form $[0 \;\; \cdots \;\; 0, \; b]$ where $b$ is a non-zero number. \par

\newpage

\textbf{Problem 1.1.28} \par
\textit{Elementary row operations on an augmented matrix never change the solution set of the associated linear system.} (T/F) \\[12pt]
This statement is $\boxed{\text{true}}$ as it is stated on the second paragraph of page 7.

\newpage

\textbf{Problem 1.1.30} \par
\textit{Two matrices are row equivalent if they have the same number of rows} (T/F) \\[12pt]
This statement is $\boxed{\text{false}}$, since two matrices are row equivalent only if there is a sequence of elementary row operations that can transform one matrix into the other. (pg 7 par 1)

\newpage

\textbf{Problem 1.2.8} \begin{align*}
    \begin{bmatrix}
        1 & 4 & 0 & 7 \\
        2 & 7 & 0 & 11
    \end{bmatrix}
    & \xrightarrow{\; R_2 \to R_2 - 2R_1 \;}
    \begin{bmatrix}
        1 & 4 & 0 & 7 \\
        0 & -1 & 0 & -3
    \end{bmatrix}
\end{align*} \begin{align*}
    & \text{Pivot Columns: } 1, 2 \\[6pt]
    & \text{Free Variables: } x_3 \\[6pt]
    & -x_2 + 0x_3 = -3 \therefore x_2 = 3 \\[6pt]
    & x_1 + 4x_2 + 0x_3 = 7 \therefore x_1 = -5 \\[6pt]
    & \boxed{\begin{cases}
        x_1 = -5 \\
        x_2 = 3 \\
        x_3 \text{ is free}
    \end{cases}}
\end{align*}

\newpage

\textbf{Problem 1.2.12} \begin{align*}
    \begin{bmatrix}
        1 & -7 & 0 & 6 & 5 \\
        0 & 0 & 1 & -2 & -3 \\
        -1 & 7 & -4 & 2 & 7
    \end{bmatrix}
    & \xrightarrow{\; R_3 \to R_1 + R_3 \;} 
    \begin{bmatrix}
        1 & -7 & 0 & 6 & 5 \\
        0 & 0 & 1 & -2 & -3 \\
        0 & 0 & 0 & 0 & 0
    \end{bmatrix}
\end{align*} \begin{align*}
    & \text{Pivot Columns: } 1, 3 \\[6pt]
    & \text{Free Variables: } x_2, x_4 \\[6pt]
    & x_1 - 7x_2 + 0x_3 + 6x_4 = 5 \therefore x_1 = 5 + 7x_2 - 6x_4 \\[6pt]
    & x_3 - 2x_4 + = -3 \therefore x_3 = -3 + 2x_4 \\[6pt]
    & \boxed{\begin{cases}
        x_1 = 5 + 7x_2 - 6x_4 \\
        x_2 \text{ is free} \\
        x_3 = -3 + 2x_4 \\
        x_4 \text{ is free}
    \end{cases}}
\end{align*}

\newpage

\textbf{Problem 1.2.20a} \par
\vspace{12pt}
a) is $\boxed{\text{consistent}}$ because the augmented matrix has no row of the form $[0 \;\; \cdots \;\; 0, \; b]$ where $b$ is a non-zero number. However, as not every column has a pivot, this system is $\boxed{\text{not unique}}$.

\bigskip

\textbf{Problem 1.2.20b} \par
\vspace{12pt}
b) is $\boxed{\text{consistent}}$ because the augmented matrix has no row of the form $[0 \;\; \cdots \;\; 0, \; b]$ where $b$ is a non-zero number. However, as not every column has a pivot, this system is $\boxed{\text{not unique}}$.

\newpage

\textbf{Problem 1.2.22} \begin{align*}
    \begin{bmatrix}
        1 & -3 & -2 \\
        5 & h & -7
    \end{bmatrix}
\end{align*} \begin{align*}
    \begin{bmatrix}
        1 & -3 & -2 \\
        5 & h & -7
    \end{bmatrix}
    & \xrightarrow{\; R_2 \to R_2 - 5R_1 \;} 
    \begin{bmatrix}
        1 & -3 & -2 \\
        0 & h + 15 & 3
    \end{bmatrix} \\[6pt]
    \begin{bmatrix}
        1 & -3 & -2 \\
        0 & h + 15 & 3
    \end{bmatrix} 
    & \xrightarrow{\; R_2 \to \frac{1}{h + 15}R_2 \;} 
    \begin{bmatrix}
        1 & -3 & -2 \\
        0 & 1 & \frac{3}{h + 15}
    \end{bmatrix} 
\end{align*}
The matrix is the augmented matrix of a consistent linear system if $\boxed{h \neq -15}$, as $h = -15$ would result in a row of the form $[0 \;\; \cdots \;\; 0, \; b]$ where $b$ is a non-zero number.

\newpage


\textbf{Problem 1.2.24a} \par
\vspace{12pt}
$\boxed{(h, k) = (9, 7)}$ will produce no solution, since the second row reduces to $\begin{bmatrix} 0 & 0 & 1 \end{bmatrix}$, which is of the form $[0 \;\; \cdots \;\; 0, \; b]$ where $b$ is a non-zero number. \par

\bigskip

\textbf{Problem 1.2.24b} \par
\vspace{12pt}
$\boxed{(h, k) = (5, 6)}$ will produce one solution, since every column that is not augmented has a pivot. \par

\bigskip

\textbf{Problem 1.2.24b} \par
\vspace{12pt}
$\boxed{(h, k) = (9, 6)}$ will produce infinite solutions, since not every column that is not augmented has a pivot. \par

\newpage

\textbf{Problem 1.2.26} \par
\textit{The echelon form of a matrix is unique} (T/F) \\[12pt]
This statement is $\boxed{\text{false}}$. Only the row reduced echelon form of a matrix is unique, as defined by Theorem 1 on page 14.

\newpage
\textbf{Problem 1.2.28} \par
\textit{The pivot positions in a matrix depend on whether row interchanges are used in the row reduction process} (T/F) \\[12pt]
This statement is $\boxed{\text{false}}$. As the book states, "the leading entries are always in the same positions in any echelon form obtained from a given matrix."

\end{document}

\maketitle  